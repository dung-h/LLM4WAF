\section{Kết luận và hướng phát triển}
\label{sec:conclusion}

\subsection{Tổng kết đóng góp}
Đồ án xây dựng một hệ thống thực nghiệm nhằm tự động hóa một phần quy trình đo lường trong môi trường WAF mô phỏng, với trọng tâm là \emph{tính tái lập} và \emph{thiết kế chỉ số}. Các đóng góp chính gồm:
\begin{itemize}
  \item Đề xuất pipeline huấn luyện ba giai đoạn (SFT \textrightarrow SFT+quan sát \textrightarrow RL) phù hợp với bối cảnh chuỗi hành động dài và tín hiệu môi trường có nhiễu.
  \item Thiết kế schema dữ liệu và tiêu chí hợp lệ để tự động hóa kiểm tra đầu ra, đồng thời đo lường được hiện tượng \emph{invalid}/\emph{refusal}.
  \item Xây dựng môi trường thực nghiệm bằng container (DVWA + WAF) nhằm tái lập thí nghiệm và phân tích tổng quát hóa theo cấu hình WAF.
  \item Báo cáo kết quả định lượng theo Pass/Invalid và thảo luận định tính về khác biệt giữa các họ mô hình.
\end{itemize}

\subsection{Bài học rút ra}
\label{subsec:lessons}
Từ quá trình triển khai, đồ án rút ra một số bài học chính:
\begin{enumerate}
  \item \textbf{Tính hợp lệ là tiền đề:} nếu đầu ra không ổn định về định dạng, mọi đo lường môi trường đều dễ sai lệch. Do đó SFT nền tảng là bước không thể thiếu.
  \item \textbf{Quan sát môi trường giúp giảm “khoảng cách RL”:} đưa tín hiệu quan sát vào SFT giúp RL ổn định hơn, đặc biệt trong không gian hành động lớn.
  \item \textbf{Alignment ảnh hưởng mạnh đến tỷ lệ invalid/refusal:} cần đo lường riêng refusal để phân biệt thất bại do môi trường và thất bại do \emph{safety policy}.
  \item \textbf{Cross-environment evaluation là bắt buộc:} một mô hình có thể thể hiện tốt ở một cấu hình nhưng suy giảm mạnh ở cấu hình khác; do đó cần đánh giá theo nhiều cấu hình và/hoặc nhiều WAF.
\end{enumerate}

\subsection{Hạn chế}
Các hạn chế chính của đồ án:
\begin{itemize}
  \item \textbf{Tài nguyên phần cứng}: dung lượng GPU hạn chế làm giảm không gian thử nghiệm hyperparameters và thuật toán RL (ví dụ PPO với batch lớn).
  \item \textbf{Đa dạng mục tiêu}: môi trường DVWA không đại diện cho các ứng dụng thực tế nhiều tầng, nhiều cơ chế xác thực/ủy quyền.
  \item \textbf{Độ đầy đủ chỉ số}: Pass Rate/Invalid Rate là cần thiết nhưng chưa đủ; cần mở rộng phân tích theo success rate và độ ổn định theo seed/cấu hình.
  \item \textbf{An toàn nội dung}: báo cáo hạn chế mô tả thao tác; điều này có thể làm giảm mức chi tiết vận hành, nhưng là cần thiết để giảm rủi ro lạm dụng.
\end{itemize}

\subsection{Hướng phát triển}
\label{subsec:future_work}
Các hướng phát triển đề xuất:
\begin{itemize}
  \item \textbf{Mở rộng môi trường}: bổ sung nhiều ứng dụng mục tiêu và cấu hình WAF khác nhau để tăng tính khái quát.
  \item \textbf{Đa chỉ số}: xây dựng bộ chỉ số đánh giá đa chiều (pass, success, validity, diversity, stability).
  \item \textbf{Thuật toán RL}: thử nghiệm PPO/kl-regularized RL và phân tích độ ổn định \cite{schulman2017ppo,ouyang2022rlhf}.
  \item \textbf{Tự động hóa kiểm thử có kiểm soát}: kết hợp cơ chế kiểm soát rủi ro (policy, redaction, sandbox) để sử dụng trong giảng dạy/đánh giá nội bộ.
  \item \textbf{Self-play đối kháng}: mở rộng sang mô hình tự chơi giữa tác nhân “đề xuất” và tác nhân “đánh giá” trong khuôn khổ có ủy quyền, tập trung vào đo lường thay vì tạo nội dung có thể lạm dụng.
\end{itemize}

\subsection{Cam kết sử dụng có trách nhiệm}
Đồ án nhấn mạnh phạm vi kiểm thử có ủy quyền và tuân thủ nguyên tắc sử dụng có trách nhiệm. Mọi kết quả và mô tả trong báo cáo nhằm phục vụ mục tiêu học thuật và cải thiện quy trình đánh giá an ninh, không nhằm hỗ trợ hành vi tấn công trái phép.
q